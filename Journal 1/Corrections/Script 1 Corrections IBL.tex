\documentclass[12pt]{article}

%----------Packages----------
\usepackage{amsmath}
\usepackage{amssymb}
\usepackage{amsthm}
%\usepackage{amsrefs}
\usepackage{dsfont}
\usepackage{mathrsfs}
\usepackage{stmaryrd}
\usepackage[all]{xy}
\usepackage[mathcal]{eucal}
\usepackage{verbatim}  %%includes comment environment
\usepackage{fullpage}  %%smaller margins
%----------Commands----------

%%penalizes orphans
\clubpenalty=9999
\widowpenalty=9999





%% bold math capitals
\newcommand{\bA}{\mathbf{A}}
\newcommand{\bB}{\mathbf{B}}
\newcommand{\bC}{\mathbf{C}}
\newcommand{\bD}{\mathbf{D}}
\newcommand{\bE}{\mathbf{E}}
\newcommand{\bF}{\mathbf{F}}
\newcommand{\bG}{\mathbf{G}}
\newcommand{\bH}{\mathbf{H}}
\newcommand{\bI}{\mathbf{I}}
\newcommand{\bJ}{\mathbf{J}}
\newcommand{\bK}{\mathbf{K}}
\newcommand{\bL}{\mathbf{L}}
\newcommand{\bM}{\mathbf{M}}
\newcommand{\bN}{\mathbf{N}}
\newcommand{\bO}{\mathbf{O}}
\newcommand{\bP}{\mathbf{P}}
\newcommand{\bQ}{\mathbf{Q}}
\newcommand{\bR}{\mathbf{R}}
\newcommand{\bS}{\mathbf{S}}
\newcommand{\bT}{\mathbf{T}}
\newcommand{\bU}{\mathbf{U}}
\newcommand{\bV}{\mathbf{V}}
\newcommand{\bW}{\mathbf{W}}
\newcommand{\bX}{\mathbf{X}}
\newcommand{\bY}{\mathbf{Y}}
\newcommand{\bZ}{\mathbf{Z}}

%% blackboard bold math capitals
\newcommand{\bbA}{\mathbb{A}}
\newcommand{\bbB}{\mathbb{B}}
\newcommand{\bbC}{\mathbb{C}}
\newcommand{\bbD}{\mathbb{D}}
\newcommand{\bbE}{\mathbb{E}}
\newcommand{\bbF}{\mathbb{F}}
\newcommand{\bbG}{\mathbb{G}}
\newcommand{\bbH}{\mathbb{H}}
\newcommand{\bbI}{\mathbb{I}}
\newcommand{\bbJ}{\mathbb{J}}
\newcommand{\bbK}{\mathbb{K}}
\newcommand{\bbL}{\mathbb{L}}
\newcommand{\bbM}{\mathbb{M}}
\newcommand{\bbN}{\mathbb{N}}
\newcommand{\bbO}{\mathbb{O}}
\newcommand{\bbP}{\mathbb{P}}
\newcommand{\bbQ}{\mathbb{Q}}
\newcommand{\bbR}{\mathbb{R}}
\newcommand{\bbS}{\mathbb{S}}
\newcommand{\bbT}{\mathbb{T}}
\newcommand{\bbU}{\mathbb{U}}
\newcommand{\bbV}{\mathbb{V}}
\newcommand{\bbW}{\mathbb{W}}
\newcommand{\bbX}{\mathbb{X}}
\newcommand{\bbY}{\mathbb{Y}}
\newcommand{\bbZ}{\mathbb{Z}}

%% script math capitals
\newcommand{\sA}{\mathscr{A}}
\newcommand{\sB}{\mathscr{B}}
\newcommand{\sC}{\mathscr{C}}
\newcommand{\sD}{\mathscr{D}}
\newcommand{\sE}{\mathscr{E}}
\newcommand{\sF}{\mathscr{F}}
\newcommand{\sG}{\mathscr{G}}
\newcommand{\sH}{\mathscr{H}}
\newcommand{\sI}{\mathscr{I}}
\newcommand{\sJ}{\mathscr{J}}
\newcommand{\sK}{\mathscr{K}}
\newcommand{\sL}{\mathscr{L}}
\newcommand{\sM}{\mathscr{M}}
\newcommand{\sN}{\mathscr{N}}
\newcommand{\sO}{\mathscr{O}}
\newcommand{\sP}{\mathscr{P}}
\newcommand{\sQ}{\mathscr{Q}}
\newcommand{\sR}{\mathscr{R}}
\newcommand{\sS}{\mathscr{S}}
\newcommand{\sT}{\mathscr{T}}
\newcommand{\sU}{\mathscr{U}}
\newcommand{\sV}{\mathscr{V}}
\newcommand{\sW}{\mathscr{W}}
\newcommand{\sX}{\mathscr{X}}
\newcommand{\sY}{\mathscr{Y}}
\newcommand{\sZ}{\mathscr{Z}}


\renewcommand{\phi}{\varphi}

\renewcommand{\emptyset}{\O}

\providecommand{\abs}[1]{\lvert #1 \rvert}
\providecommand{\norm}[1]{\lVert #1 \rVert}


\providecommand{\x}{\times}




\providecommand{\ar}{\rightarrow}
\providecommand{\arr}{\longrightarrow}





%----------Theorems----------

\newtheorem{theorem}{Theorem}[section]
\newtheorem{proposition}[theorem]{Proposition}
\newtheorem{lemma}[theorem]{Lemma}
\newtheorem{corollary}[theorem]{Corollary}

\theoremstyle{definition}
\newtheorem{definition}[theorem]{Definition}
\newtheorem*{definition*}{Definition}
\newtheorem{nondefinition}[theorem]{Non-Definition}
\newtheorem{exercise}[theorem]{Exercise}



\numberwithin{equation}{subsection}


%----------Title-------------
\title{Sheet 1: Basics}
\author{John Lind}

\begin{document}

\pagestyle{plain}


%%---  sheet number for theorem counter

\begin{center}
{\large MATH 161, SHEET 1: SETS, FUNCTIONS and CARDINALITY} \\ 
\vspace{.2in}  
%John Boller, Diane Herrmann, Paul Sally, September 1, 2012
Jeffrey Zhang IBL Script 1 Corrections (6 November 2013)
\end{center}

\bigskip \bigskip

\medskip

\textbf{Exercise 1.16}
Let $A=\{1,2,3\}$.  Identify $\wp(A)$ by explicitly listing its elements.
$\wp(A) = $ \{{$\emptyset$, \{{1\}},\{{2\}},\{{3\}},\{{1,2\}},\{{2,3\}},\{{1,3\}},\{{1,2,3\}}\}

\medskip

\textbf{Lemma 1.25}
Suppose that $f \colon A \rightarrow B$ is bijective.  
Then there exists a bijection $g \colon B \rightarrow A$.

\begin {proof}
We define $g(B) = \{a \in A \mid f(a) \in B\}$. We know that $f$ is bijective, so $f$ is injective and it follows by Definition 1.20 that for every $a \in A$ there exists a unique $b \in B$ such that $f(a) = b$. Thus, we know that $g$ satisfies the definition of a function (Definition 1.17). Let $x, x' \in X$ such that $g(x) = g(x')$. Let $g(x) = a$ for some $a \in A$, then $f(a) = x$. $g(x) = g(x')$ so it follows that $g(x') = a$, $f(a) = x'$. Then $x = x'$, so $g$ is injective (Definition 1.20). Let $a \in A$ and $f(a) = b$. We have shown that $f$ is injective, so we know that there exists a unique $b \in B$ such that $f(a) = b$. It follows then that $\exists g(b) = a$ for some $b \in B$. So we can say that $\forall a \in A$ such that $f(a) = b$, $\exists b \in B$ such that $g(b) = a$. Then by definition 1.20, $g$ is surjective.
\end {proof}

\textbf{Lemma 1.29} Let $A$, $B$, and $C$ be sets and suppose that there is a bijective correspondence between $A$ and $B$ and a bijective correspondence between $B$ and $C$.  Then there is a bijective correspondence between $A$ and $C$.
\begin{proof}
Let $f: $ $A \to B$, $g: $ $B \to C$. We know $f$ and $g$ are bijections. Suppose $c \in C$, then $\exists b \in B$ such that $g(b) = c$ by Definition 1.20. By the same definition we also know that $\forall b  \in B$ $\exists a \in A$ such that $f(a) = b$. It follows then that for $c \in C$, $\exists a \in A$ such that $g(f(a)) = c$. We define a function $h$ such that $h = g \circ f$, so $h: $ $A \to C$ and $h$ is surjective.
Suppose $a, a' \in A$ and $h(a) = h(a')$. It follows then that $g(f(a)) = g(f(a'))$. We know by Definition 1.20 that $\forall b, b' \in B$, if $g(b) = g(b')$ then $b = b'$. So we know that $f(a) = f(a')$. By the same definition, we know that $\forall a, a' \in A$, if $f(a) = f(a')$ then $a = a'$. Thus, if $h(a) = h(a')$ then $a = a'$, so h is injective. Because $h$ is injective and surjective, $h$ is bijective.
\end{proof}

\textbf{Exercise 1.34}
Let $A$ and $B$ be two finite sets.  Then $|A\times B|=|A|\cdot |B|$.

\begin{proof}
Let $|A| = m, |B| = n$.
We let the proposition $P(n): |A \times B| = |A| \cdot |B|$. To prove the base case, we let $n = 1$. It follows then that $|A| = m$, $|B| = 1$, so $|A \times B| = m$, which is obviously true.
Let the inductive hypothesis be that $|A \times B|  = |A| \cdot |B|$ for two sets $A,B$. We define two sets $A, B$ such that $|A| = m$ and $|B| = n + 1$. It follows then that $|A| \cdot |B| = m(n+1)$. Let $B = C \cup D$ for two disjoint sets $C,D$ such that $|C| = n$ and $|D| = 1$. We know that the set $|A \times B| = \{(a,b)|a \in A,b \in B\}$, and we know that $|A| + |(C \cup A)| + |C| = |A| \cdot |C| = m * n$ using the inductive hypothesis. It follows then that $|A \times B| = m * (n + 1)$. Thus, using induction we know that $|A \times B| = |A| \cdot |B|$ for two finite sets $A, B$.
\end{proof}

\end{document}
