\documentclass[12pt]{article}


%----------Packages----------
\usepackage{amsmath}
\usepackage{amssymb}
\usepackage{amsthm}
%\usepackage{amsrefs}
\usepackage{dsfont}
\usepackage{mathrsfs}
\usepackage{stmaryrd}
\usepackage[all]{xy}
\usepackage[mathcal]{eucal}
\usepackage{verbatim}  %%includes comment environment
\usepackage{fullpage}  %%smaller margins
%----------Commands----------

%%penalizes orphans
\clubpenalty=9999
\widowpenalty=9999





%% bold math capitals
\newcommand{\bA}{\mathbf{A}}
\newcommand{\bB}{\mathbf{B}}
\newcommand{\bC}{\mathbf{C}}
\newcommand{\bD}{\mathbf{D}}
\newcommand{\bE}{\mathbf{E}}
\newcommand{\bF}{\mathbf{F}}
\newcommand{\bG}{\mathbf{G}}
\newcommand{\bH}{\mathbf{H}}
\newcommand{\bI}{\mathbf{I}}
\newcommand{\bJ}{\mathbf{J}}
\newcommand{\bK}{\mathbf{K}}
\newcommand{\bL}{\mathbf{L}}
\newcommand{\bM}{\mathbf{M}}
\newcommand{\bN}{\mathbf{N}}
\newcommand{\bO}{\mathbf{O}}
\newcommand{\bP}{\mathbf{P}}
\newcommand{\bQ}{\mathbf{Q}}
\newcommand{\bR}{\mathbf{R}}
\newcommand{\bS}{\mathbf{S}}
\newcommand{\bT}{\mathbf{T}}
\newcommand{\bU}{\mathbf{U}}
\newcommand{\bV}{\mathbf{V}}
\newcommand{\bW}{\mathbf{W}}
\newcommand{\bX}{\mathbf{X}}
\newcommand{\bY}{\mathbf{Y}}
\newcommand{\bZ}{\mathbf{Z}}

%% blackboard bold math capitals
\newcommand{\bbA}{\mathbb{A}}
\newcommand{\bbB}{\mathbb{B}}
\newcommand{\bbC}{\mathbb{C}}
\newcommand{\bbD}{\mathbb{D}}
\newcommand{\bbE}{\mathbb{E}}
\newcommand{\bbF}{\mathbb{F}}
\newcommand{\bbG}{\mathbb{G}}
\newcommand{\bbH}{\mathbb{H}}
\newcommand{\bbI}{\mathbb{I}}
\newcommand{\bbJ}{\mathbb{J}}
\newcommand{\bbK}{\mathbb{K}}
\newcommand{\bbL}{\mathbb{L}}
\newcommand{\bbM}{\mathbb{M}}
\newcommand{\bbN}{\mathbb{N}}
\newcommand{\bbO}{\mathbb{O}}
\newcommand{\bbP}{\mathbb{P}}
\newcommand{\bbQ}{\mathbb{Q}}
\newcommand{\bbR}{\mathbb{R}}
\newcommand{\bbS}{\mathbb{S}}
\newcommand{\bbT}{\mathbb{T}}
\newcommand{\bbU}{\mathbb{U}}
\newcommand{\bbV}{\mathbb{V}}
\newcommand{\bbW}{\mathbb{W}}
\newcommand{\bbX}{\mathbb{X}}
\newcommand{\bbY}{\mathbb{Y}}
\newcommand{\bbZ}{\mathbb{Z}}

%% script math capitals
\newcommand{\sA}{\mathscr{A}}
\newcommand{\sB}{\mathscr{B}}
\newcommand{\sC}{\mathscr{C}}
\newcommand{\sD}{\mathscr{D}}
\newcommand{\sE}{\mathscr{E}}
\newcommand{\sF}{\mathscr{F}}
\newcommand{\sG}{\mathscr{G}}
\newcommand{\sH}{\mathscr{H}}
\newcommand{\sI}{\mathscr{I}}
\newcommand{\sJ}{\mathscr{J}}
\newcommand{\sK}{\mathscr{K}}
\newcommand{\sL}{\mathscr{L}}
\newcommand{\sM}{\mathscr{M}}
\newcommand{\sN}{\mathscr{N}}
\newcommand{\sO}{\mathscr{O}}
\newcommand{\sP}{\mathscr{P}}
\newcommand{\sQ}{\mathscr{Q}}
\newcommand{\sR}{\mathscr{R}}
\newcommand{\sS}{\mathscr{S}}
\newcommand{\sT}{\mathscr{T}}
\newcommand{\sU}{\mathscr{U}}
\newcommand{\sV}{\mathscr{V}}
\newcommand{\sW}{\mathscr{W}}
\newcommand{\sX}{\mathscr{X}}
\newcommand{\sY}{\mathscr{Y}}
\newcommand{\sZ}{\mathscr{Z}}


\renewcommand{\phi}{\varphi}

\renewcommand{\emptyset}{\O}

\providecommand{\abs}[1]{\lvert #1 \rvert}
\providecommand{\norm}[1]{\lVert #1 \rVert}


\providecommand{\ar}{\rightarrow}
\providecommand{\arr}{\longrightarrow}

\renewcommand{\_}[1]{\underline{ #1 }}


\DeclareMathOperator{\ext}{ext}



%----------Theorems----------

\newtheorem{theorem}{Theorem}[section]
\newtheorem{proposition}[theorem]{Proposition}
\newtheorem{lemma}[theorem]{Lemma}
\newtheorem{corollary}[theorem]{Corollary}


\newtheorem*{axiom4}{Axiom 4}


\theoremstyle{definition}
\newtheorem{definition}[theorem]{Definition}
\newtheorem{nondefinition}[theorem]{Non-Definition}
\newtheorem{exercise}[theorem]{Exercise}
\newtheorem{remark}[theorem]{Remark}
\newtheorem{warning}[theorem]{Warning}
\newtheorem{examples}[theorem]{Examples}
\newtheorem{example}[theorem]{Example}



\numberwithin{equation}{subsection}


%----------Title-------------
%\title{Sheet 5: ContinuOUS FUNCTIONS}
%\author{John Lind}

\begin{document}

\begin{center}
{\large SHEET 9: SEQUENCES and LIMITS} \\ 
\vspace{.2in}  
Jeffrey Zhang
\end{center}

\bigskip \bigskip


%%---  sheet number for theorem counter
\setcounter{section}{9}   


We will now work with the real numbers $\bbR$ instead of an arbitrary continuum $C$.  
Accordingly, let us now use the standard notation $(a, b)$ for the region 
$\_{ab} = \{ x \in \bbR : a < x < b \}$.  Even though the notation is the same, this is \emph{not} the same object as the ordered pair $(a, b)$.


\begin{definition}  A \emph{sequence} (of real numbers) is a function $a \colon \bbN \arr \bbR$.
\end{definition}

\noindent By setting $a_n = a(n)$, we can think of a sequence $a$ as a list $a_1, a_2, a_3, \dotsc$ of real numbers.  We use the notation $(a_n)_{n=1}^{\infty}$ for such a sequence, or if there is no possibility of confusion, we sometimes abbreviate this and write simply $(a_n)$.  %More generally, we also use the term sequence to refer to a function defined on $\{n\in \bbN: n\geq n_0\}$ for any fixed $n_0\in \bbN$.  We write $(a_n)_{n = n_{0}}^{\infty}$ for such a sequence.

\begin{definition}  
We say that a sequence $(a_n)$ \emph{converges} to a point $p \in \bbR$ if, for every region~$R$ containing $p$, there exists $N\in \bbN$ such that if $n\geq N$, then $a_n\in R$.
If $(a_n)$ does not converge to any point, we say
that the sequence \emph{diverges}.
\end{definition}

\begin{exercise}
Show that if a sequence $(a_n)$ converges to $p$, then any region containing $p$ contains all
but finitely many terms in the sequence.
\end{exercise}

\begin{proof}
Let $(a_n)$ converge to $p$, and $R$ be an arbitrary region such that $p \in R$. Then we know by 9.2 that there exists $N \in \bbN$ such that if $n \geq N$, $a_n \in R$. $R$ contains all $a_n \in R$ such that $n \geq N$, so $R$ contains at all but at most $N - 1$ terms in the sequence. $N - 1$ is finite, so $R$ contains all but finitely many terms in the sequence.
\end{proof}

\begin{exercise}  Which of the following sequences converge?  Which diverge?
If one converges, what does it converge to?  If one diverges, what can you say about the 
nature of its divergence?

\centerline{
\vbox{\hbox{(a)\quad $a_n = 5$}
\hbox{(b)\quad $a_n = n$}}\hfil
\vbox{\hbox{(c)\quad  $a_n = 1/n$}
\hbox{(d)\quad  $a_n = (-1)^n$}}\hfil
\vbox{\hbox{(e)\quad  $a_n = (-1)^n\cdot n$}
\hbox{(f)\quad  $a_n = (-1)^n\cdot\frac{1}{n}$}}}
\end{exercise}

\begin{proof}
(a) $a_n = 5$ converges to $5$. \newline
(b) $a_n = n$ diverges to infinity. \newline
(c) $a_n = \frac{1}{n}$ converges to $0$. \newline
(d) $a_n = (-1)^n$ diverges, alternating between -1 and 1. \newline
(e) $a_n = (-1)^n \cdot n$ diverges. \newline
(f) $a_n = (-1)^n \cdot \frac{1}{n}$ converges to $0$.
\end{proof}

\begin{theorem}  Suppose that $(a_n)$ converges both to $p$ and to $p'$.
Then $p = p'$.
\end{theorem}

\begin{proof}
Let $(a_n)$ converge to $p$ and $p'$ such that $p \not = p'$. Then we know that we can choose two regions $R, R'$ around $p,p'$ respectively such that $R \cap R' = \emptyset$. Then we know by Exercise 9.3 that $R, R'$ each contain all but finitely many terms of the sequence. So it follows then that because $R$ and $R'$ are disjoint, $R$ and $R'$ can each contain only finitely many terms in the sequence. This is a contradiction, because it means that $a_n$ has finitely many terms, and by Definition 9.1 we have that $a_n$ has infinitely many terms. So we have a contradiction, and $p$ must be equal to $p'$. 
\end{proof}

\begin{definition}
If a sequence $(a_n)$ converges to $p\in\bbR$, we call $p$ the \emph{limit} of $(a_n)$ and write
\[
\lim_{n \rightarrow \infty} a_n = p.
\]
\end{definition}



\begin{definition}
Let $A$ and $B$ be ordered sets. A function $f:A\to B$ is said to be \emph{increasing} if $a<a'$ implies $f(a)<f(a')$, and \emph{decreasing} if $a<a'$ implies $f(a)>f(a')$.
\end{definition}

\begin{definition}  Let $(a_n)$ be a sequence.  A \emph{subsequence} of $(a_n)$ is a sequence 
$b\colon \bbN \rightarrow \bbR$ defined by the composition $b = a \circ i$, where 
$i \colon \bbN \rightarrow \bbN$ is an increasing function.  
If $(a_n)$ has a subsequence with limit $p$, we call $p$ a \emph{subsequential limit} of $(a_n)$.
\end{definition}

\noindent  If we let $n_k = i(k) \in \bbN$, we can write $b_k = a_{n_k}$, so that $(b_k)$ is the 
sequence $b_1, b_2, b_3, \dotsc$, which is equal to the sequence
$a_{n_1}, a_{n_2}, a_{n_3}, \dotsc, $ where $n_1 < n_2 < n_3 < \dotsm$.

\begin{theorem}  If $(a_n)$ converges to $p$, then so do all of its subsequences.
\end{theorem}

\begin{proof}
Let $b_n$ be a subsequence of $a_n$ such that $b_k = a_{i(k)}$ for $i$ an increasing function. Suppose $a_n$ converges to $p \in \bbR$, then by 9.2 we know that for any region $R$ such that $p \in R$, there exists $N \in \bbN$ such that for all $n \geq N$, $a_n \in R$. Then choose $m > N$ such that $m = i(k_0)$ for $k_0 \in \bbN$. Then for all $k \geq k_0$, we know that $i(k) \geq i(k_0)$ as $i$ is increasing. It follows then that $k \geq k_0$, so we know that $i(k) \geq m > N$. $a_{i(k)} = b_k$ and $b_k \in R$ for all $k \geq k_0$, so we have by 9.2 that $b_n$ converges to $p$ .
\end{proof}

\begin{exercise}  Construct a sequence with two subsequential limits.  Construct a sequence with infinitely many subsequential limits.
\end{exercise}

\begin{proof}
The sequence $a_n = (-1)^n$ has two subsequential limits, as we have the subsequence defined by $b_n = 1$ and the subsequence defined by $c_n = -1$. $b_n$ converges to $1$ and $c_n$ converges to $-1$. \newline
The sequence $a_n = 1,1,2,1,2,3,1,2,3,4,1,2,3,4,5, \dotsm, n$. Then we can find a subsequence $b_n$ such that $b_n = 1, b_n = 2, b_n = 3, \dotsm, b_n = n$. These subsequences then converge to $1,2,3,\dotsm,n$ respectively, so we have infinitely many subsequential limits.
\end{proof}

Let $p \in \bbR$ and for each natural number $k \geq 1$, define $R_{k}$ to be the region $(p - \tfrac{1}{k}, p + \tfrac{1}{k})$.  

\begin{lemma}\label{lemma1}
The $R_k$ form a descending collection of regions $R_1 \supset R_2 \supset R_3 \supset \dotsm$ whose intersection is the point $p$:
\[
\bigcap_{k \geq 1} R_k = \{ p \}.
\]
\end{lemma}

\begin{proof}
We know that $p - \frac{1}{k} < p - \frac{1}{k+1}$ and that $p + \frac{1}{k+1} < p + \frac{1}{k}$ so it folllows that $(p - \frac{1}{k}, p + \frac{1}{k}) \supset (p - \frac{1}{k+1}, p + \frac{1}{k+1})$. It follows then that $R_1 \supset R_2 \supset R_3 \dotsm$. \newline
We know that $p \in \bigcap_{k \geq 1} R_k$, so we now show that $\bigcap_{k \geq 1} R_k = \{ p \}$. Let $q \in \bigcap_{k \geq 1} R_k$ such that $q \not = p$. Then we know that $q > p$ or $q < p$. If $q > p$, then we have that $q - p \in \bbR$ and $q - p > 0$ so we have by 8.25 that $q - p > \frac{1}{k} > 0$ for some $k \in \bbN$. Then it follows that $q > p + \frac{1}{k}$ for $k \in \bbN$, so $q \not \in (p - \frac{1}{k}, p + \frac{1}{k})$. This is a contradiction as we now have that $q \not \in \bigcap_{k \geq 1} R_k$. The same result can be shown for the case of $q < p$ without loss of generality. So we have that $\bigcap_{k \geq 1} R_k = \{ p \}$.
\end{proof}

\begin{lemma}\label{lemma2}  
Let $R$ be a region containing $p$.  Then there exists a natural number $N$ such that $R_k \subset R$ for all $k \geq N$.
\end{lemma}

\begin{proof}
Let $S = (a,b)$ be a region containing $p$ for $a,b \in \bbR$. Then we have that $a < p$, so $p - a > 0$ and we now that $p - a \in \bbR$. It follows then by 8.25 that there exists $M$ such that $p - a > \frac{1}{M}$ for $M \in \bbN$. Note then that we have $a < p - \frac{1}{M}$ for $M \in \bbN$. We now consider $p + \frac{1}{M}$ and take two cases, where $p + \frac{1}{M} \geq b$ and where $p + \frac{1}{M} < b$. \newline 
Consider $p + \frac{1}{M} \geq b$. $b > p$, so we know there exists $N \in \bbN$ such that $b - p > \frac{1}{N}$ by 8.25. So it follows then that $b > p + \frac{1}{N}$. Then we have $p + \frac{1}{M} \geq b > p + \frac{1}{N}$. It follows then that $p + \frac{1}{M} > p + \frac{1}{N}$, so $p - \frac{1}{M} < p - \frac{1}{N}$. Recall $p - \frac{1}{M} > a$, so then we have $a < p - \frac{1}{N}$ and that $p + \frac{1}{N} < b$, so we know that $R_N \subset S$ and by 9.11 we know that the collection of regions is descending, so we have that $R_k \subset S$ for all $k \geq N$. \newline
Consider $p + \frac{1}{M} < b$, then we have that $a < p - \frac{1}{M}$ and that $p + \frac{1}{M} < b$, so we have that $R_M \subset S$ and similarly then that $R_k \subset S$ for all $k \geq M$.
\end{proof}

\begin{theorem}  Let $A \subset \bbR$.  Then $p \in \overline{A}$ if and only if there exists a sequence $(a_n)$, with each $a_n \in A$, that converges to $p$.
\end{theorem}

\begin{proof} 
Let $A \subset \bbR$ and suppose $p \in \overline{A}$. Then we know that for all regions $R$ such that $p \in R$, $R \cap A \not = \emptyset$. We also know that for a region $R$ such that $p \in R$, by 9.12 there exists $N \in \bbN$ such that $R_k \subset R$ for all $k \geq N$. So we choose $a_k \in R_k$, then we have that $a_k \in R_k \subset R$, so we know then by 9.2 that $a_k$ converges to $p$. \newline
Suppose there exists a sequence $a_n$ such that for $a_n \in A$, $a_n$ converges to $p$. By 9.3, we know that for any $p \in R$ for a region $R$, $R$ contains all but finitely many terms in the sequence. So we have that for any $R$ such that $p \in R$, $R \cap A \not = \emptyset$. It follows then that $p \in \overline{A}$.
\end{proof}

\begin{definition}  A sequence $(a_n)$ is \emph{bounded} if its range $\{a_n\mid n\in\bbN\}$ is bounded.
\end{definition}

\begin{theorem}  Every convergent sequence is bounded.
\end{theorem}

\begin{proof}
Let $a_n$ be a convergent sequence such that $a_n$ converges to $p$ for $p \in \bbR$. Let $R = (a,b)$ for $a,b \in \bbR$ be a region, then we know there exists $N \in \bbN$ such that for all $n \geq N$, $a_n \in (a,b)$. Thus, we have that for all $n \geq N$, $a_n < b$. We have for $\{a_1, a_2, \dotsm, a_(n-1)\}$ that there exists a greatest element $a_k$. If $b > a_k$, then $b$ is an upper bound of $a_n$ and we have that $a_n$ is bounded. If $a_k > b$, then $a_k \geq a_n$ and we have that $\max(a_k, b)$ is an upper bound of $a_n$. So we know that $a_n$ is bounded above. Without loss of generality an analogous proof may be used to show that convergent sequences are bounded below.
\end{proof}

The converse is not true, but there are two important partial converses.

\begin{theorem}[Monotone Bounded Sequence Theorem]
Every bounded increasing sequence converges to its range's supremum.
Every bounded decreasing sequence converges to its range's infimum.
\end{theorem}

\begin{proof}
Let $a_n$ be a bounded increasing sequence. Then by 8.22, we know that $s = \sup(a_n)$ exists. Note that by definition of $\sup(a_n)$, we know that for $\epsilon > 0$, $s - \epsilon < a_M$ for some $a_M \in a_n$. We know that $a_n$ is increasing, so for $m > M$, $a_m > a_M$. It follows then that $s - a_m < s - a_M$, so $s - a_m < s - a_M < \epsilon$ (recall that $s - \epsilon < a_M$) and so we have that for all $n \geq M$, $a_n \in R$. Thus we know that for a region $S$ such that $s = \sup(a_n) \in S$, we have that some $a_n \in S$ so $a_n$ converges to $s = \sup(a_n)$. Similarly, it can be shown that a bounded decreasing sequence converges to its range's infimum.
\end{proof}

\begin{theorem}  Every bounded sequence has a convergent subsequence. 
\emph{Hint: sometimes this is called the Bolzano--Weierstrass theorem.} 
\end{theorem}

\begin{proof}
Let $a_n$ be bounded, then we know that the range of $a_n$ is bounded. Let $A$ be the range of $a_n$. Then we know that $A$ is bounded and we consider two cases, where $A$ is finite and where $A$ is infinite. \newline
We consider the case where $A$ is finite, then we know that for any $n \in bbN$, $a_n$ has finitely many possible values. Let $t$ be the finite number of possible values. Then the set $\{a_n, \dotsm, a_{n+t}\}$ has $t + 1$ values in it so there must be at least one repeated value. We know that the sequence is infinite, so the cycles of $\{a_n, \dotsm, a_{n+t}\}$ are repeated infinitely. Thus, some number $p$ must be repeated infinitely many times, so we take the subsequence $\{p, \dotsm, p\}$ which we know is convergent. \newline
Now consider the case where $A$ is infinite, then we know by Bolzano-Weierstrauss that a limit point $p$ of $A$ exists. So for all regions $R$ containing $p$, we know that $R \cap (A \setminus \{p\}) \not = \emptyset$. Then we know that for a region $R_k = (p - \frac{1}{k}, p + \frac{1}{k})$, there exists some $a_{n_k} \in R_k$. Let $R$ be an arbitrary region containing $p$, then by 9.12, we know that there exists $N \in \bbN$ such  that $R_k \subset R$ for all $k \geq \bbN$. Thus we know  that there exists some $k \geq N$ such that $a_{n_k} \in R_k \subset R$, so we have that there exists some $a_{n_k} \in R$. So by 9.2, we know there exists a convergent subsequence.
\end{proof}

Mathematicians often use the letters $\delta$ and $\epsilon$ to denote small positive numbers.

\begin{lemma}  Let $R$ be a region containing the point $a$.  Then there exists a number $\delta > 0$ such that $(a - \delta, a + \delta) \subset R$.
\end{lemma}

\begin{proof}
Let $R$ be a region containing $a$. We write $R = (m,n)$ for $m,n \in \bbR$. Then we let $\delta = \min(|a - m|, |a-n|)$. It follows then that $\delta \leq |a-m|$ and $\delta \leq |a-n|$, so we know that $a - \delta \geq m$ and that $a + \delta \leq n$. Then we have that $(a - \delta, a + \delta) \subset R$. 
\end{proof}

\begin{definition}
The {\em absolute value} of a real number $x$ is the non-negative number $\abs{x}$ defined by:
\[
\abs{x} = \begin{cases}
x \quad &\text{if $x \geq 0$,} \\
-x \quad &\text{if $x < 0$.} 
\end{cases}
\]
\end{definition}

\begin{exercise}  Prove that:
\[
(a - \delta, a + \delta) = \{ x \in \mathbb{R} : \abs{x - a} < \delta \}.
\]
\end{exercise}

\begin{proof}
Let $p \in (a - \delta, a + \delta)$. Then we have $a - \delta < p < a + \delta$, and so $|p - a| < \delta$, so $p \in \{ x \in \mathbb{R} : \abs{x - a} < \delta \}$. \newline
Let $p \in \{ x \in \mathbb{R} : \abs{x - a} < \delta \}$, then we know $a - \delta < p < a + \delta$ so $p \in (a - \delta, a +\delta)$. \newline
So we have $(a - \delta, a + \delta) = \{ x \in \mathbb{R} : \abs{x - a} < \delta \}$.
\end{proof}
We deduce a more concrete characterization of continuity at a point (Definition~5.11).

\begin{theorem}
Let $A\subset\bbR$, $f:A\to\bbR$ and $a\in A$.
Then $f$ is continuous at $a$ if and only if the following condition holds: for every $\epsilon > 0$, there exists $\delta > 0$ such that
\[
\text{ if\quad $x\in A$\quad and\quad $\abs{x - a} < \delta$,\qquad then\quad $\abs{f(x) - f(a)} < \epsilon$.}
\]
\end{theorem}

\begin{proof}
We first show the forward direction. \newline
 Let $\epsilon > 0$, then let a region $R = (f(a) - \epsilon, f(a) + \epsilon)$. We are given $f$ is continuous at $a$, so we know by 5.11 that there exists a region $S$ such that for $a \in S$, $f(S \cap A) \subset R$. $a \in S$, so it follows then by 9.18 that there exists $\delta  > 0$ such that $(a - \delta, a + \delta) \subset S$. Let $x \in A$ and $x \in (a - \delta, a + \delta)$, then it follows that $|x - a| < \delta$ by 9.20. $x \in (S \cap A)$, so $f(x) \in f(S \cap A)$. Then we have that $f(x) \in f(S \cap A) \subset (f(a) - \epsilon, f(a) + \epsilon)$, so it follows again by 9.20 that $|f(x) - f(a)| < \epsilon$. So we have that the forward direction holds. \newline
We now show the reverse direction. \newline
Let $R$ be a region such that $f(a) \in R$. We express $R = (m,n)$ for $m,n \in \bbR$. Then we define $\epsilon = \min(|f(a) - m|, |f(a) - n|)$. Note that $(f(a) - \epsilon, f(a) + \epsilon) \subset R$. $\epsilon > 0$, so we are given that there exists $\delta > 0$ such that $|x - a| < \delta$ for all $x \in A$. Then we let the region $S = (a - \delta, a + \delta)$. Let $x \in S \cap A$, then we know that $|x - a| < \delta$ and $x \in A$ so $|f(x) - f(a)| < \epsilon$. It follows then by 9.20 that $f(x) \in (f(a) - \epsilon, f(a) + \epsilon)$, so $f(x) \in R$ because $(f(a) - \epsilon, f(a) + \epsilon) \subset R$. So we have that $f(S \cap A) \subset R$, and thus $f$ is continuous at $a$ by 5.11. So we have that the reverse direction holds. \newline
\end{proof}

\begin{exercise}
\begin{enumerate}
\item Let $a,b\in \bbR$ and let $f:\bbR\longrightarrow \bbR$ be given by $f(x)=ax+b.$ Show that $f$ is continuous at every $x\in\bbR.$
\item Let $f:\bbR\longrightarrow\bbR$ be given by $f(x)=\begin{cases} 1 & \text{if  }x\neq 0\\ 0 & \text{if }x=0.\end{cases}$ Show that $f$ is not continuous at $0.$
\end{enumerate}
\end{exercise}

\begin{proof}
1. Let $\epsilon > 0$ be arbitrary. Let $x \in \bbR$ be arbitrary. Then we have $f(x) = ax + b$ for $a,b \in \bbR$. Let $p \in \bbR$, and let $\delta = \frac{\epsilon}{|a|}$ such that $|x - p| < \delta$. It follows then that $|a(x-p)| < \epsilon$. Note that $|f(x) - f(a)| = |ax + b - ap - b| = |ax - ap| = |a(x-p)|$, so we have that $|f(x) - f(a)| < \epsilon$. So we know then that $f$ is continuous at $p$ for every $p \in \bbR$. \newline

2. Assume that $f$ is continuous at 0. Let $\epsilon = 0.5$ and $a = 0$. Then we have that there exists $\delta$ such that $|x - a| < \delta$. So we have $|x - 0| < \delta$. We choose $x \in \bbR$ such that $x \not = 0$. We assume $f$ is continuous at $0$, so we know that $|x - 0| < \delta$ implies that $|f(x) - f(0)| < \epsilon$. However, we have $x \not = 0$ so $f(x) = 1$ and thus $|f(x) - f(0)| = |1 - 0| = 1 \not < 0.5$, so we have a contradiction. Thus, $f$ must not be continuous at $0$.
\end{proof}

\end{document}